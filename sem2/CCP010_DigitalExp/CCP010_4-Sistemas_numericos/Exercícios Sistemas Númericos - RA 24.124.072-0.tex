\documentclass{article}[12pt]
\usepackage[a4paper, margin=1in]{geometry}
\usepackage{amsmath}

\begin{document}

\section*{Exercícios}

\begin{enumerate}
    \item Os computadores utilizam o sistema binário, ou de base 2, que é um sistema de numeração em que todas as quantidades se representam com base nos números 0 e 1.
    \begin{enumerate}
        \item Como será a representação binária do número 2024 em um computador? \\
        \textbf{R.:} 0111 1110 1000
        
        \item Como será a representação desse mesmo número nas bases octal e hexadecimal? \\
        \textbf{R.:} 
        \begin{itemize}
            \item \textbf{Octal}: 3750
            \item \textbf{Hexadecimal}: 7E8
        \end{itemize}
        
        \item Se os computadores trabalham representando informações com números binários, por que estudar as bases octal e hexadecimal? \\
        \textbf{R.:} As bases octal e hexadecimal são diferentes formas de representar a base binária de forma compactada, facilitando o entendimento. Atualmente, a base octal é menos utilizada que a hexadecimal, porém ambas ainda são válidas.
    \end{enumerate}
    
    \item Realize as seguintes conversões:
        \begin{enumerate}
            \item 325 para binário \\
            \textbf{R.:} 0001 0100 0101$_2$ \\
            \textbf{Passo a passo:} Divida 325 por 2 até o quociente ser 0, anotando os restos:
            \begin{itemize}
                \item 325 ÷ 2 = 162, \phantom{0}resto 1
                \item 162 ÷ 2 = \phantom{0}81, \phantom{0}resto 0
                \item \phantom{0}81 ÷ 2 = \phantom{0}40, \phantom{0}resto 1
                \item \phantom{0}40 ÷ 2 = \phantom{0}20, \phantom{0}resto 0
                \item \phantom{0}20 ÷ 2 = \phantom{0}10, \phantom{0}resto 0
                \item \phantom{0}10 ÷ 2 = \phantom{00}5, \phantom{0}resto 0
                \item \phantom{00}5 ÷ 2 = \phantom{00}2, \phantom{0}resto 1
                \item \phantom{00}2 ÷ 2 = \phantom{00}1, \phantom{0}resto 0
                \item \phantom{00}1 ÷ 2 = \phantom{00}0, \phantom{0}resto 1
            \end{itemize}
            Os restos lidos de baixo para cima formam o número binário: 101000101$_2$.

            \item 10100$_2$ para decimal \\
            \textbf{R.:} 20 \\
            \textbf{Passo a passo:} Multiplique cada dígito pelo valor da posição correspondente (potência de 2):
            \begin{itemize}
                \item $1 \times 2^4$ = \phantom{0}16
                \item $0 \times 2^3$ = \phantom{00}0
                \item $1 \times 2^2$ = \phantom{00}4
                \item $0 \times 2^1$ = \phantom{00}0
                \item $0 \times 2^0$ = \phantom{00}0
            \end{itemize}
            Somando os valores: 16 + 4 = 20.

            \item 455$_8$ para hexadecimal \\
            \textbf{R.:} 12D$_{16}$ \\
            \textbf{Passo a passo:} Converta primeiro de octal para binário e depois de binário para hexadecimal:
            \begin{itemize}
                \item 4$_8$ = 100$_2$
                \item 5$_8$ = 101$_2$
                \item 5$_8$ = 101$_2$
            \end{itemize}
            Juntando os binários: 100101101$_2$. Agrupe em blocos de 4 bits: 0001 0010 1101$_2$.
            Converta cada bloco para hexadecimal:
            \begin{itemize}
                \item 0001 = 1
                \item 0010 = 2
                \item 1101 = D
            \end{itemize}
            Portanto, 455$_8$ = 12D$_{16}$.

            \item ABAE$_{16}$ para decimal \\
            \textbf{R.:} 43.950 \\
            \textbf{Passo a passo:} Multiplique cada dígito pelo valor da posição correspondente (potência de 16):
            \begin{itemize}
                \item A = 10; B = 11; A = 10; E = 14
                \item $10 \times 16^3$ = \phantom{0}40.960
                \item $11 \times 16^2$ = \phantom{00}2.816
                \item $10 \times 16^1$ = \phantom{000}160
                \item $14 \times 16^0$ = \phantom{0000}14
            \end{itemize}
            Somando os valores: 40.960 + 2.816 + 160 + 14 = 43.950.

            \item 10111000$_2$ para hexadecimal \\
            \textbf{R.:} B8$_{16}$ \\
            \textbf{Passo a passo:} Agrupe em blocos de 4 bits: 1011 1000$_2$.
            Converta cada bloco para hexadecimal:
            \begin{itemize}
                \item 1011 = B
                \item 1000 = 8
            \end{itemize}
            Portanto, 10111000$_2$ = B8$_{16}$.

            \item 23,1875 para binário \\
            \textbf{R.:} 0001 0111,0011$_2$ \\
            \textbf{Passo a passo:} Converta a parte inteira e a parte fracionária separadamente: \\
                \textbf{Parte inteira:}
                \begin{itemize}
                    \item 23 ÷ 2 = \phantom{0}11, resto 1
                    \item 11 ÷ 2 = \phantom{00}5, resto 1
                    \item \phantom{0}5 ÷ 2 = \phantom{00}2, resto 1
                    \item \phantom{0}2 ÷ 2 = \phantom{00}1, resto 0
                    \item \phantom{0}1 ÷ 2 = \phantom{00}0, resto 1
                \end{itemize}
                Lendo os restos de baixo para cima: 10111$_2$.

                \textbf{Parte fracionária:}
                \begin{itemize}
                    \item $0.1875           \times 2 = 0.375$, parte inteira 0
                    \item $0.375\phantom{0} \times 2 = 0.75\phantom{0}$, parte inteira 0
                    \item $0.75\phantom{00} \times 2 = 1.5\phantom{00}$, parte inteira 1
                    \item $0.5\phantom{000} \times 2 = 1.0\phantom{00}$, parte inteira 1
                \end{itemize}
            Lendo as partes inteiras: 0011$_2$.
            Portanto, 23,1875 = 10111,0011$_2$.

            \item 0,1 para binário \\
            \textbf{R.:} 0000,1111 1111 1111$_2$ \\
            \textbf{Passo a passo:} Multiplique a parte fracionária por 2 repetidamente:
            \begin{itemize}
                \item $0.1 \times 2 = 0.2$, parte inteira 0
                \item $0.2 \times 2 = 0.4$, parte inteira 0
                \item $0.4 \times 2 = 0.8$, parte inteira 0
                \item $0.8 \times 2 = 1.6$, parte inteira 1
                \item $0.6 \times 2 = 1.2$, parte inteira 1
                \item $0.2 \times 2 = 0.4$, parte inteira 0
                \item $0.4 \times 2 = 0.8$, parte inteira 0
                \item $0.8 \times 2 = 1.6$, parte inteira 1
                \item $0.6 \times 2 = 1.2$, parte inteira 1
                \item $0.2 \times 2 = 0.4$, parte inteira 0
                \item $0.4 \times 2 = 0.8$, parte inteira 0
                \item $0.8 \times 2 = 1.6$, parte inteira 1
                \item $0.6 \times 2 = 1.2$, parte inteira 1
                \item $0.2 \times 2 = 0.4$, parte inteira 0
                \item $0.4 \times 2 = 0.8$, parte inteira 0
                \item $0.8 \times 2 = 1.6$, parte inteira 1
            \end{itemize}
            Portanto, 0.1 = 0.00011001100110011...$_2$ (repetindo).

            \item 11101,01$_2$ para decimal \\
            \textbf{R.:} 29,25 \\
            \textbf{Passo a passo:} Converta a parte inteira e a parte fracionária separadamente: \\
                \textbf{Parte inteira:}
                \begin{itemize}
                    \item $1 \times 2^4$ = 16
                    \item $1 \times 2^3$ = \phantom{0}8
                    \item $1 \times 2^2$ = \phantom{0}4
                    \item $0 \times 2^1$ = \phantom{0}0
                    \item $1 \times 2^0$ = \phantom{0}1
                \end{itemize}
                Somando os valores: 16 + 8 + 4 + 0 + 1 = 29.

                \textbf{Parte fracionária:}
                \begin{itemize}
                    \item $0 \times 2^{-1}$ = 0
                    \item $1 \times 2^{-2}$ = 0,25
                \end{itemize}
                Somando os valores: 0 + 0,25 = 0,25.

            Portanto, 11101,01$_2$ = 29,25.

            \item 678,25 para binário \\
            \textbf{R.:} 0010 1010 0110,01$_2$ \\
            \textbf{Passo a passo:} Converta a parte inteira e a parte fracionária separadamente: \\
                \textbf{Parte inteira:}
                \begin{itemize}
                    \item 678 ÷ 2 = 339, resto 0
                    \item 339 ÷ 2 = 169, resto 1
                    \item 169 ÷ 2 = \phantom{0}84, resto 1
                    \item \phantom{0}84 ÷ 2 = \phantom{0}42, resto 0
                    \item \phantom{0}42 ÷ 2 = \phantom{0}21, resto 0
                    \item \phantom{0}21 ÷ 2 = \phantom{0}10, resto 1
                    \item \phantom{0}10 ÷ 2 = \phantom{00}5, resto 0
                    \item \phantom{00}5 ÷ 2 = \phantom{00}2, resto 1
                    \item \phantom{00}2 ÷ 2 = \phantom{00}1, resto 0
                    \item \phantom{00}1 ÷ 2 = \phantom{00}0, resto 1
                \end{itemize}
                Lendo os restos de baixo para cima: 1010100110$_2$.

                \textbf{Parte fracionária:}
                \begin{itemize}
                    \item $0.25 \times           2 = 0.5$, parte inteira 0
                    \item $0.5\phantom{0} \times 2 = 1.0$, parte inteira 1
                \end{itemize}
                Lendo as partes inteiras: 01$_2$.
            Portanto, 678,25 = 1010100110,01$_2$.

            \item 11100,011$_2$ para decimal \\
            \textbf{R.:} 28,375 \\
            \textbf{Passo a passo:} Converta a parte inteira e a parte fracionária separadamente: \\
                \textbf{Parte inteira:}
                \begin{itemize}
                    \item $1 \times 2^4$ = 16
                    \item $1 \times 2^3$ = \phantom{0}8
                    \item $1 \times 2^2$ = \phantom{0}4
                    \item $0 \times 2^1$ = \phantom{0}0
                    \item $0 \times 2^0$ = \phantom{0}0
                \end{itemize}
                Somando os valores: 16 + 8 + 4 + 0 + 0 = 28.

                \textbf{Parte fracionária:}
                \begin{itemize}
                    \item $0 \times 2^{-1}$ = 0
                    \item $1 \times 2^{-2}$ = 0,25
                    \item $1 \times 2^{-3}$ = 0,125
                \end{itemize}
                Somando os valores: 0 + 0,25 + 0,125 = 0,375.

            Portanto, 11100,011$_2$ = 28,375.

            \item A64$_{16}$ para binário \\
            \textbf{R.:} 1010 0110 0100$_2$ \\
            \textbf{Passo a passo:} Converta cada dígito hexadecimal para binário:
            \begin{itemize}
                \item A = 1010
                \item 6 = 0110
                \item 4 = 0100
            \end{itemize}
            Portanto, A64$_{16}$ = 1010 0110 0100$_2$.

            \item D52$_{16}$ para decimal \\
            \textbf{R.:} 3.410 \\
            \textbf{Passo a passo:} Multiplique cada dígito pelo valor da posição correspondente (potência de 16):
            \begin{itemize}
                \item D = 13; 5 = 5; 2 = 2
                \item $13           \times 16^2 =           3.328$
                \item $5\phantom{0} \times 16^1 = \phantom{0.0}80$
                \item $2\phantom{0} \times 16^0 = \phantom{0.00}2$
            \end{itemize}
            Somando os valores: 3.328 + 80 + 2 = 3.410.
        \end{enumerate}
    
    \item A maioria das pessoas pode contar até 10 nos dedos das mãos. Porém, cientistas da computação podem fazer melhor:
    \begin{enumerate}
        \item Se você considerar cada dedo como um bit binário, com o dedo estendido indicando 1 e o dedo recolhido indicando 0, até quanto você pode contar usando as mãos? \\
        \textbf{R.:} $2^{10}-1 = 1.023$
        
        \item Se você considerar o dedão da mão esquerda como sendo um bit de sinal para números de complemento de dois, qual é faixa de números que é possível ser expressa dessa forma? \\
        \textbf{R.:} -256 até 255
    \end{enumerate}
\end{enumerate}

\end{document}