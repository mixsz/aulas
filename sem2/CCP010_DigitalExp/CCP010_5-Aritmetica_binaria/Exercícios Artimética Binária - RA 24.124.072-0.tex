\documentclass[a4paper, 12pt]{article}
\usepackage[a4paper, margin=1in]{geometry}
\usepackage{amsmath}

\begin{document}

\section*{Exercícios}

\subsection*{1. Efetue as seguintes somas ou subtrações em binário. Verifique os resultados convertendo os números e fazendo os cálculos também na base decimal.}
\begin{enumerate}
    \item[a)] \(        \phantom{0001\ } 1010 \phantom{,0100}+ \phantom{\ 0000\ }1011 \phantom{,0010}        = \quad      \phantom{0000\ } \mathbf{\ 0001\ 0101}\)
    \item[b)] \(        \phantom{0001\ } 1111 \phantom{,0100}+ \phantom{\ 0000\ }0011 \phantom{,0010}        = \quad      \phantom{0000\ } \mathbf{\ 0001\ 0010}\)
    \item[c)] \(         \phantom{0001\ }1011,1101           + \phantom{\ 0000\ 00}11,1 \phantom{000}        = \quad            \phantom{0000\ } \mathbf{\ 1111,0101}\)
    \item[d)] \(         \phantom{0001\ 000}0,1011           + \phantom{\ 0000\ 000}0,1111                   = \quad    \phantom{0000\ 0000\ } \mathbf{\ \ 0001,1010}\)
    \item[e)] \(                   1001\ 1011 \phantom{,0100} +           \ 1001\ 1101 \phantom{,0010}       = \quad                   \mathbf{0010\ 0011\ 1000}\)
    \item[f)] \(         \phantom{0001\ }1010,01\phantom{00}  + \phantom{\ 0001\ 00}10,111\phantom{0}        = \quad     \phantom{0000\ 0000\ }\mathbf{\ \ 1101,001}\)
    \item[g)] \(                   1000\ 1111 \phantom{,0100} +           \ 0101\ 0001 \phantom{,0010}       = \quad      \phantom{0010\ } \mathbf{\ 1101\ 0000}\)
    \item[h)] \(                   1100\ 1100 \phantom{,0100} +           \ 0011\ 0111 \phantom{,0010}       = \quad                   \mathbf{0001\ 0000\ 0011}\)
    \item[i)] \(         \phantom{0001\ }1010 \phantom{,0100} - \phantom{\ 0000\ }0111       \phantom{,0010} = \quad    \phantom{\ 0001\ 0000\ } \mathbf{\ 0011}\)
    \item[j)] \(                   0010\ 1010 \phantom{,0100} -           \ 0010\ 0101 \phantom{,0010}       = \quad    \phantom{\ 0001\ 0000\ } \mathbf{\ 0101}\)
    \item[k)] \(         \phantom{0001\ }1111,010\phantom{0}  -\phantom{\ 0000 }\ 1000,001\phantom{0}        = \quad    \phantom{\ 0001\ 0000\ } \mathbf{\ 0111,001}\)
    \item[l)] \(                   0001\ 0011  \phantom{,0100} -           \ 0000\ 0110  \phantom{,0010}     = \quad    \phantom{\ 0001\ 0000\ } \mathbf{\ 1101}\)
    \item[m)] \(                   1110\ 0010  \phantom{,0100} -           \ 0101\ 0001  \phantom{,0010}     = \quad                   \mathbf{0010\ 1001\ 0001}\)
\end{enumerate}

\subsection*{2. Represente cada um dos números decimais seguintes no sistema do complemento de 2. Use um total de 8 bits, incluindo o bit de sinal.}

\begin{enumerate}
    \item[a)] \(+  \phantom{1}32 = \mathbf{0010\ 0000}\)
    \item[b)] \(-  \phantom{1}14 = \mathbf{1111\ 0010}\)
    \item[c)] \(+  \phantom{1}63 = \mathbf{0011\ 1111}\)
    \item[d)] \(-            104 = \mathbf{1001\ 1000}\)
    \item[e)] \(+            127 = \mathbf{0111\ 1111}\)
    \item[f)] \(-            127 = \mathbf{1000\ 0001}\)
    \item[g)] \(-  \phantom{12}1 = \mathbf{1111\ 1111}\)
    \item[h)] \(-            128 = \mathbf{1000\ 0000}\)
    \item[i)] \(+            169 = \mathbf{1010\ 1001}\)
    \item[j)] \( \phantom{- 12}0 = \mathbf{0000\ 0000}\)
    \item[k)] \(+  \phantom{1}84 = \mathbf{0101\ 0100}\)
    \item[l)] \(+  \phantom{12}3 = \mathbf{0000\ 0011}\)
\end{enumerate}

\subsection*{3. Realize as seguintes operações no sistema do complemento de 2. Use 8 bits (incluindo o de sinal) para cada número. Verifique os resultados convertendo o resultado binário de volta para decimal.}

\begin{enumerate}
    \item[a)] Subtraia \(+16\) de \(+17\)
    \begin{itemize}
        \item Representação em binário:
        \begin{align*}
            +17: & \quad (00010001) \\
            +16: & \quad (00010000)
        \end{align*}
        \item Subtração:
        \[
        (00010001 - 00010000 = \mathbf{00000001})
        \]
        \item Resultado em decimal: \(1\)
    \end{itemize}

    \item[b)] Some \(+19\) a \(-24\)
    \begin{itemize}
        \item Representação em binário:
        \begin{align*}
            +19: & \quad (00010011) \\
            -24: & \quad (11101000) \quad \text{(complemento de dois)}
        \end{align*}
        \item Soma:
        \[
        (00010011 + 11101000 = \mathbf{11111111})
        \]
        \item Resultado em decimal: \(-5\)
    \end{itemize}

    \item[c)] Some \(-48\) a \(-80\)
    \begin{itemize}
        \item Representação em binário:
        \begin{align*}
            -48: & \quad (11010000) \quad \text{(complemento de dois)} \\
            -80: & \quad (10110000) \quad \text{(complemento de dois)}
        \end{align*}
        \item Soma:
        \[
        (11010000 + 10110000 = \mathbf{110000000}) \quad \text{(overflow)}
        \]
        \item Resultado em decimal: \(128\) \quad \text{(overflow)}
    \end{itemize}

    \item[d)] Subtraia \(-36\) de \(-15\)
    \begin{itemize}
        \item Representação em binário:
        \begin{align*}
            -15: & \quad (11110001) \quad \text{(complemento de dois)} \\
            -36: & \quad (11011100) \quad \text{(complemento de dois)}
        \end{align*}
        \item Subtração:
        \[
        (11110001 - 11011100 = \mathbf{00010101})
        \]
        \item Resultado em decimal: \(21\)
    \end{itemize}

    \item[e)] Some \(+17\) a \(-17\)
    \begin{itemize}
        \item Representação em binário:
        \begin{align*}
            +17: & \quad (00010001) \\
            -17: & \quad (11101111) \quad \text{(complemento de dois)}
        \end{align*}
        \item Soma:
        \[
        (00010001 + 11101111 = \mathbf{00000000})
        \]
        \item Resultado em decimal: \(0\)
    \end{itemize}

    \item[f)] Subtraia \(-17\) de \(-17\)
    \begin{itemize}
        \item Representação em binário:
        \begin{align*}
            -17: & \quad (11101111) \quad \text{(complemento de dois)}
        \end{align*}
        \item Subtração:
        \[
        (11101111 - 11101111 = \mathbf{00000000})
        \]
        \item Resultado em decimal: \(0\)
    \end{itemize}

    \item[g)] Some \(+68\) a \(+45\)
    \begin{itemize}
        \item Representação em binário:
        \begin{align*}
            +68: & \quad (01000100) \\
            +45: & \quad (00101101)
        \end{align*}
        \item Soma:
        \[
        (01000100 + 00101101 = \mathbf{01110001})
        \]
        \item Resultado em decimal: \(113\)
    \end{itemize}

    \item[h)] Subtraia \(-50\) de \(+77\)
    \begin{itemize}
        \item Representação em binário:
        \begin{align*}
            +77: & \quad (01001101) \\
            -50: & \quad (11001110) \quad \text{(complemento de dois)} \\
            +50: & \quad (00110010)
        \end{align*}
        \item Subtração:
        \[
        (01001101 + 00110010 = \mathbf{01111111})
        \]
        \item Resultado em decimal: \(127\)
    \end{itemize}
\end{enumerate}

\subsection*{4. Realize as seguintes operações de números binários e verifique os resultados fazendo também as contas na base decimal.}

\begin{enumerate}
    \item[a)] \(111\phantom{10} \times \phantom{1}101 = \mathbf{0010\ 0011}\)
    \item[b)] \(1011\phantom{0} \times           1011 = \mathbf{0111\ 1001}\)
    \item[c)] \(10110           \times \phantom{1}111 = \mathbf{1001\ 1010}\)
    \item[d)] \(111\phantom{10} \div   \phantom{11}10 = \mathbf{0011},\ \text{resto}\ \mathbf{1}\)
    \item[e)] \(1010\phantom{1} \div             1010 = \mathbf{0001}\)
\end{enumerate}

\end{document}